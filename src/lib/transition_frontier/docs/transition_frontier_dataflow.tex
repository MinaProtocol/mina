\documentclass{article}
\usepackage{calc}
\usepackage{tikz}
\usepackage{trace}
\usepackage{forloop}
\usetikzlibrary{arrows.meta}
\usetikzlibrary{calc}

\newlinechar=`\^^n

%\newcommand{\forrange}[5][1]{%
%  \setcounter{#2}{#3}%
%  \ifthenelse{\value{#2} < \value{#4}}%
%    {%
%      #5%
%      \addtocounter{#2}{#1}%
%      \forrange[#1]{#2}{\value{#2}}{#4}{#5}%
%    }%
%    {%
%    }%
%}%

\def\tfhistory(#1)#2(#3)#4(#5){
  %\forrange{n}{1}{#1}{
  %  \pgfmathtruncatemacro{\ni}{\value{n}}
  %  \node[draw,circle] (xx-tfh-\ni) at ($(#3)+(\value{n},0)$) {};
  %}
  %\draw[thick] ($(xx-tfh-1)+(-0.25,0.25)$) rectangle ($(xx-tfh-\ni)+(0.25,-0.25)$);
  \newcounter{n}
  \newcounter{max}
  \setcounter{max}{#3}
  \newcommand{\name}[1]{#1-##1}
  %\node[draw,circle] (\name{1}) at (#5) {};
  \forloop{n}{1}{\value{n} < \value{max}}{
    \pgfmathtruncatemacro{\ni}{\value{n}+1}
    \node[draw,circle] (\name{\ni}) at ($(#5)+(\value{n},0)$) {};
    \draw[thick] (\name{\value{n}}) -- (\name{\ni});
  }
  \addtocounter{n}{1}
  \draw[thick] ($(\name{1})+(-0.25,0.25)$) rectangle ($(\name{#3})+(0.25,-0.25)$);
}%

\begin{document}
\begin{figure}
\begin{tikzpicture}
  \node[draw,circle] (tfh-1) at (0,0) {};
  \node[draw,circle] (tfh-2) at (1,0) {};
  \node[draw,circle] (tfh-3) at (2,0) {};
  \node[draw,circle] (tfh-4) at (3,0) {};
  \draw[thick] (tfh-1) -- (tfh-2);
  \draw[thick] (tfh-2) -- (tfh-3);
  \draw[thick] (tfh-3) -- (tfh-4);
  \draw[thick] ($(tfh-1)+(-0.25,0.25)$) rectangle ($(tfh-4)+(0.25,-0.25)$);

  %\newcounter{n}
  %\newcounter{max}
  %\setcounter{max}{4+1}
  %\pgfmathtruncatemacro{\ni}{\value{n}}
  %\forloop{n}{1}{\value{n} < \value{max}}{
  %  \node[draw,circle] (tfh2-\ni) at ($(0,-4)+(\value{n},0)$) {{}};
  %}
  \traceon
  \tfhistory (xx-tfh) (4) (0,-4);
  \traceoff
\end{tikzpicture}
\end{figure}
\end{document}
